\documentclass[letterpaper, 11pt]{article}
    \usepackage[margin=0.9in]{geometry}
    \usepackage{graphicx, hyperref, booktabs}
    % rubber: set program xelatex
    \usepackage[stix]{bccho}

% Begin document

\begin{document}
    \begin{center}
        \large
        \textsc{\textbf{ELE 302 -- Independent Project Final Proposal}} \vspace{5pt}

        \normalsize
        TJ Smith \hspace{1cm} Byung-Cheol Cho \\
        (Bench 207) \vspace{5pt}

        \emph{Due March 31, 2017}
        \normalsize
    \end{center}

\section{Overview}
\begin{itemize}
    \item \textbf{Objective:} To build a robot/car capable of playing basic ping pong
    \item \textbf{General principles:}
        \begin{itemize}
            \item We will use a camera (or two for stereoscopic vision if necessary) to track a ping pong ball and estimate its trajectory.
            \item We will then move the robot to the required location (ideally using omni wheels or Mecanum wheels) and hit the ball back over the net.
            \item We will use a combination of dead reckoning and a grid laid on the floor to determine the position of the robot
        \end{itemize}
\end{itemize}

\section{Progress steps and checkpoints}
\subsection*{Hardware}
\begin{enumerate}[label=\textbf{(\arabic*)}]
    \item Move around with new drive system (build H-bridges and other relevant circuit boards as necessary)
    \item Track position of robot with grid and dead reckoning (with accelerometer and gyro)
    \item Set up interfaces between camera, accelerometer/gyroscope, primary processor (either a Raspberry Pi or an Arduino) and PSoC
    \item Set up paddle for hitting ping pong ball
\end{enumerate}

\subsection*{Software}
\begin{enumerate}[label=\textbf{(\arabic*)}]
    \item Set up feedback control to execute movement commands
    \item Determine the location of a static ping pong ball based on camera $x$/$y$ coordinates and ball size
    \item Predict the trajectory of a moving ping pong ball based on camera measurements
    \item Calculate position to move car to in order to hit ping pong ball
\end{enumerate}

\subsection*{Intermediate goals}
\begin{enumerate}[label=\textbf{(\arabic*)}]
    \item Execute movement commands and track position
    \item Track ping pong ball and rotate to face ball
    \item Predict ball landing position $X$ and move to point $X$ in a non-time-dependent manner
    \item Track ball and move so that the ball hits the robot's paddle after one bounce
    \item Track ball and move to hit the ball back over the net after one bounce
\end{enumerate}

\section{Parts list}
\begin{itemize}
    \item Omnidirectional drive, ideally able to move sufficiently fast (2-3 feet/second)
        \begin{itemize}
            \item If possible, the omni chassis from last year, or an equivalent one from the same vendor. We may need to gear up the motor if necessary
            \item Four H-bridges (lots of transistors)
        \end{itemize}
    \item Raspberry Pi 3 (or perhaps an Arduino?)
    \item Pixy CMUcam5 (maybe two for stereo vision)
    \item Accelerometer and gyroscope
    \item Light sensor (either C-Cam-2A or photoresistor with bright LEDs)
    \item Paddle (any hard, flat surface that a ping pong ball can bounce off)
\end{itemize}

\end{document}
