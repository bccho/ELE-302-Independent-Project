\documentclass[letterpaper, 11pt]{article}
    \usepackage[margin=0.9in]{geometry}
    \usepackage{graphicx, hyperref, booktabs}
    % rubber: set program xelatex
    \usepackage[stix]{bccho}

% Begin document

\begin{document}
    \begin{center}
        \large
        \textsc{\textbf{ELE 302 -- Independent Project Initial Proposal}} \vspace{5pt}

        \normalsize
        TJ Smith \hspace{1cm} Byung-Cheol Cho \\
        (Bench 207) \vspace{5pt}

        \emph{Due March 17, 2017}
        \normalsize
    \end{center}

\section{Overview}
\begin{itemize}
    \item \textbf{Objective:} To build a robot/car capable of playing basic ping pong
    \item \textbf{General principles:}
        \begin{itemize}
            \item We will use a camera (or two for stereoscopic vision if necessary) to track a ping pong ball and estimate its trajectory.
            \item We will then move the robot to the required location (ideally using omni wheels or Mecanum wheels) and hit the ball back over the net.
            \item We will use a combination of dead reckoning and a grid laid on the floor to determine the position of the robot
        \end{itemize}
\end{itemize}

\section{Goals and checkpoints}
\subsection*{Hardware}
\begin{enumerate}[label=\textbf{(\arabic*)}]
    \item Moving with omniwheels
    \item Tracking position
    \item Camera interfacing
    \item Movement commands
\end{enumerate}

\subsection*{Software}
\begin{enumerate}[label=\textbf{(\arabic*)}]
    \item Determine the location of a static ping pong ball based on $x$/$y$ coordinates and ball size
    \item Predict the trajectory
    \item Calculate position to move to
\end{enumerate}
% TODO: Including natural stopping points

\section{Progress steps}
% TODO: General idea of how to do it

\section{Parts list}
% TODO: rough parts list

\end{document}
